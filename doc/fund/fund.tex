\documentclass[arabic]{article}

\usepackage{arabtex}
\usepackage[utf8]{inputenc}
\usepackage[LFE,LAE]{fontenc}
\usepackage[arabic]{babel}

\setcode{utf8}

\title{\Huge{طلب تمويل}}
\date{\vspace{-5ex}}

\begin{document}

\maketitle


\section{اسم المشروع باللغة العربية}

مشروع ميكروكمبيوتر قائم على عمارة
\textLR{MIPS}
باستخدام مصفوفة البوابات القابلة للبرمجة
\textLR{FPGA}.

\section{اسم المشروع باللغة الإنجليزية}

\textLR{FPGA Microcomputer based on MIPS architecture}.

\section{نبذة مختصرة عن المشروع}

يهدف المشروع إلى تصميم وتنفيذ جهاز كمبيوتر تعليمي مدمج باستخدام تقنية
\textLR{FPGA}.
والغرض الرئيسي من الكمبيوتر هو توفير منصة تعليمية لطلاب قسم الحاسب والنظم
بكلية الهندسة بحيث
يمكن استخدامها في معمل عمارة الحاسب وبرمجة النظم.
يقوم معالج الكمبيوتر على معمارية
\textLR{MIPS}
ومن المقرر تزويده بنظام تشغيل متعدد الأغراض.
يوفر الكمبيوتر مجموعة من الأدوات والتجارب المعملية التي يستطيع من خلالها الطلاب
تطبيق الطرق الهندسية التي يتم تدريسها في المقررات المرتبطة بتنظيم الحاسبات.
\section{المعدات المطلوبة}

\begin{itemize}
\item
لوحة
\textLR{FPGA}
من طراز
\textLR{Altera DE1-Soc Board}.

السعر:
\I{1365}
جنيه مصري.

تكاليف الشحن:
\I{800}
جنيه مصري.

\end{itemize}


\end{document}
